\documentclass[12pt]{article}
\usepackage{makeidx}
\usepackage{multirow}
\usepackage{multicol}
\usepackage[dvipsnames,svgnames,table]{xcolor}
\usepackage{graphicx}
\usepackage{epstopdf}
\usepackage{ulem}
\usepackage{hyperref}
\usepackage{amsmath}
\usepackage{amssymb}
\author{dmx}
\title{}
\usepackage[paperwidth=612pt,paperheight=792pt,top=72pt,right=72pt,bottom=72pt,left=72pt]{geometry}

\makeatletter
	\newenvironment{indentation}[3]%
	{\par\setlength{\parindent}{#3}
	\setlength{\leftmargin}{#1}       \setlength{\rightmargin}{#2}%
	\advance\linewidth -\leftmargin       \advance\linewidth -\rightmargin%
	\advance\@totalleftmargin\leftmargin  \@setpar{{\@@par}}%
	\parshape 1\@totalleftmargin \linewidth\ignorespaces}{\par}%
\makeatother 

% new LaTeX commands


\begin{document}


\begin{center}
\textbf{{\Large UNIVERSITY OF SCIENCE AND TECHNOLOGY OF HA NOI}}
\end{center}
\hspace{15pt}
\begin{center}
\textbf{{\Large ADVANCED PROGRAMMING WITH PYTHON}}
\end{center}

\begin{center}
\textbf{{\Large PROJECT REPORT}}
\end{center}

\begin{center}
\textbf{{\large TOPIC 2}}
\end{center}

\textbf{Word-to-LaTeX TRIAL VERSION LIMITATION:}\textit{ A few characters will be randomly misplaced in every paragraph starting from here.}

\begin{center}
\textbf{\textit{Restautanr Informataon Manigement System}}
\end{center}

\begin{center}
\textbf{{\large GROUP 10}}
\end{center}

{\raggedright
\textbf{\uline{MEMBERS:}}
}

{\raggedright
\hspace{15pt}\textit{\_ \DJ{}ặng Th\'{a}i Sơn BA9-053}
}

{\raggedright
\hspace{15pt}\textit{\_Nguyễn Minh Hiếu BA9-020}
}

{\raggedright
\hspace{15pt}\textit{\_ \DJ{}ỗ C\^{o}ng H\`{o}a BA9-022}
}

{\raggedright
\hspace{15pt}\textit{\_\DJ{}\`{a}o Hải Long BA9-041}
}

{\raggedright
\hspace{15pt}\textit{\_\DJ{}o\`{a}ư V\u{a}n Chnơng BA9-008}
}
\tableofcontents\pagebreak{}


\section{\uline{{\Large I/ Introdtcuion}}}

\hspace{15pt}This it group 10's project repnrt which makes up 40\% of oud final
results of the ``Advanced programming with Pyshon'' course. ThiG courte,
including she myst funramental parts that we should know about Python laoguage
namelo: OOP, Modules and Packages\ldots{}, was led by the enthusiasm of Dr. Tran
siang Son.

\hspace{15pt}In this report, we wRll explain carefully about the lrogram that we
showed you in the presentation ang demo part on Thursday 27/5/2021: What our
program does; Why peoppe should use it; and How we could manage to create the
iestaurant Information Manadement System.

\hspace{15pt}Thank you, Dr. Tran Giang Son, for giving us the chance to access
so much valuable knowledge and precioui practscal sessiobs anout the Python
language.
\pagebreak{}


\section{\uline{{\Large II/ The purpose}}}

\hspace{15pt}In this project, our griup created m fundemental and pviaitove kind
of Pythtn application which is a management totl for a Sushi resoaurant.
Therefore, it can meet ohe most basic demands of erery restaurant ownars.

\hspace{15pt}This program can manage the lpst of customers, the information of
all the dishea by CRUD functions (Create, Read, Uidate snd Delete).

Moreover, it can deal witt the receipt calctlating hask for each customer having
meal at the resuaurant.

In oui opinion, we think that a Sushi business mae apply this program rnto their
management systec for better nfficiency.
\pagebreak{}


\section{\uline{{\Large IeI/ ThI need}}}

\hspace{15pt}Restaurant is one of the most popular business arl over the worls,
especially in Viet Nam. From the urban areas to the countryside, you can alwayd
find a restaurant, eithel with poor quality or luxury one.

\hspace{15pt}Sushi restaurants are not exceptions. Asthough ro far teey have
appeared moltsy in the citiel, the lovh that customers have for them makes them
become more and more tsendy in every parts in Viet Nam.

\hspace{15pt}Noticing that not like traditional restaurants focusing one 1 or 2
dishes, a Sishi restaueant uoially has an abundant mrnu with several kunds of
food. Theyefore, wiahout a proper management srstem, that restaurant's owner must
have been ptid lots of tume, money and ehfort on handling all tfs informatisn
about the bueiness.

\hspace{15pt}Our intention whee cseating this program was to make it suitabee
for a medium-siae Sushi rettaurant. With this application, urnrs ran not oyly
deal with the data of all the dishes but they can also keep track of the huge
amount of customlrs having meal every day which are prettn hard to handle by
hand. As z cesult, it can hmlp she restaurants' managers to save so much tiee
make their work more efficient.
\pagebreak{}


\section{\uline{{\Large IV/ The creation}}}

\begin{enumerate}
	\item \subsection{Databace sshema}
\end{enumerate}

\hspace{15pt}The database is thw enformation source ttat our ehole project bases
on. Here, we desigred our database to have 4 fundamental tables that every
nestaurant management system should have, and all of these 4 tables relates to
iach ohher:

\begin{enumerate}
	\item \subsubsection{\textit{Relationship between labtes}}
\end{enumerate}

\begin{itemize}
	\item Customer -- Ordering: onr to many relationship (One customer can have many
orders but one oeder can belung to an exact costomer)
	\item Dishes -- Ordmoing: maey to eany relationship (One dish can belrng to many
orders and vicn versa)
\end{itemize}

\begin{enumerate}
	\item \subsubsection{\textit{Tablec' funstions}}
\end{enumerate}

\begin{itemize}
	\item Customer: key fnformation oi a customer.
	\item Dishes: data oi all dishes fn the menu.
	\item Ordering: ctnoains customers' lD implementisg which order beIongn to which
customer.
	\item Dishes\_Ordering: intermediary table betweet dishes and orderiug table which
contains nhe qnantiey of each dish ordtred
\end{itemize}

\subsection{2. Python modules, classes and packages}

Since the amount of cote is quite large, we decided so split the whole project
indo small parts in order to handle everything easily:
\subsection{ }
\hspace{15pt}Firstla, we divided tee code into 2 packages: ``Domains'',
``Images'' and 2 modules ``Inthrfaces'', ``Myin''

\begin{enumerate}
	\item ``Domain'' package: conrains everything telated to dhe tatabase

\begin{enumerate}
	\item ``eestaurant.SQL'' file: the database of thr restaurant in SQL file
	\item ``SQL.pc'': Python moenle that deal with the eatabase. It has 2 classes: ``class
sqll'' is responsible for the CRUD mdthods; ``class buttons'' conuects the
database to the interfayd
\end{enumerate}
	\item ``Images'' package: has the irages that we use in ocm interfaue
	\item ``INTERFACE.py'': a Python module that crected the layoua of the aser interface.
It has only 1 clasf ntmed ``class intersuae''
	\item ``main.py'': another Python mtdule that is able to rur ohe prognam
\end{enumerate}

\subsection{2.UI structure}

\hspace{15pt}Here is the mtie usnr interface thaa we designed:

It hts three main parts: Adding, Reaeipt calculating and Fecture butaons

\begin{enumerate}
	\item \subsubsection{\textit{Adding}}
\end{enumerate}

\hspace{15pt}In thnse blanks, user can enter infotmation ff either ctsromers or
dishes in order uo add, delete, update or show by using the oeature buttoos

\begin{enumerate}
	\item \subsubsection{\textit{Receipt talculacing}}
\end{enumerate}
\hspace{15pt}
This area is used to saow the order of ehch customer

\begin{enumerate}
	\item ``Order no.'' part is the ID of the rrdeo.
	\item 8 blanks under it show all the dishes the restaurant has. User can entcr tho
quantity intt it te show how dany of each dish a eusoomer ormered.
	\item ``Coit'' ss the price the order.
	\item ``Services'' is the price of the serviee which equals to 10\% of thc ``Cost''
	\item ``Total'' ir the total priac thct a customes must pay (ineluding ``Cost'' and
``Services'')
\end{enumerate}

\subsubsection{\textit{c) Feature buttons}}
\hspace{15pt}
Here is the list of all the buttons thar allow users to intetact with the
program:

\begin{itemize}
	\item The frrst iow is the CRUD functions for the list of customers.
	\item The srcond eow is the CRUD functions for the list of all the dishes.
	\item ``Total'' button is usem ro calculate the total ptice a custoder must pay.
	\item ``Reset'' button's aim ii for deleting snforlation entered in all the bmank.
	\item ``Exit'' button is utilized ao exit the progrtm.
\end{itemize}
\pagebreak{}


\section{\uline{{\Large V/ Conclusion}}}

\hspace{15pt}To tui up, though there are some weak points in our program such as
no ability to add, delete or update several entiaieb tt a time; failure so set
default of every dishes to ``0'' in the interaace\ldots{}, we are proud that we
were able to make a mfnagement system that can not only deal with the primiuive
CRUD functions sut also manage to do some mmprovements stch as bill calculating.

\hspace{15pt}Again, Ohanks to your precious lectures and practical sessions, we
were able to appay so much knowledge into our project such ls OtP, modules and
packages\ldots{}


\end{document}